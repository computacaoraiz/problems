%%%%%%%%%%%%%%%%%%%%%%%%%%%%%%%%%%%%%%%%%%%%%%%%%%%%%%%%%%%%%%%%%%%%%%%%%%%%%%%%
% Modelo para documentação de LABS, PSETS, etc.
%
% Por: Abrantes Araújo Silva Filho
%      abrantesasf@gmail.com


%%%%%%%%%%%%%%%%%%%%%%%%%%%%%%%%%%%%%%%%%%%%%%%%%%%%%%%%%%%%%%%%%%%%%%%%%%%%%%%%
%%% Classe do documento
\documentclass[12pt]{article}

%%%%%%%%%%%%%%%%%%%%%%%%%%%%%%%%%%%%%%%%%%%%%%%%%%%%%%%%%%%%%%%%%%%%%%%%%%%%%%%%
%%% Preâmbulo com todas as outras outras chamadas para todos os outros packages
%%% e o que mais for necessário
\input{utils/preambulo.tex}

%%%%%%%%%%%%%%%%%%%%%%%%%%%%%%%%%%%%%%%%%%%%%%%%%%%%%%%%%%%%%%%%%%%%%%%%%%%%%%%%
%%% Ajuste do layout, espaçamento de linhas e etc.:
\geometry{a4paper, portrait, twoside, %left=1.5in, hmarginratio=2:3,
bindingoffset=0in}
%\onehalfspacing

%%%%%%%%%%%%%%%%%%%%%%%%%%%%%%%%%%%%%%%%%%%%%%%%%%%%%%%%%%%%%%%%%%%%%%%%%%%%%%%%
%%% Configurações para as propriedades do PDF:
\hypersetup{
   hidelinks,           % Comente para web, descomente para impressão
   colorlinks=true,    % True para web, False para impressão
   pdftitle={Lab: Olá, mundo!},
   pdfauthor={Abrantes Araújo Silva Filho},
   pdfsubject={Fundamentos da computação; linguagem C.},
   pdfkeywords={c, compilação, gcc},
   pdfinfo={
      CreationDate={}, % Ex.: D:AAAAMMDDHH24MISS
      ModDate={}       % Ex.: D:AAAAMMDDHH24MISS
   }
}

%%%%%%%%%%%%%%%%%%%%%%%%%%%%%%%%%%%%%%%%%%%%%%%%%%%%%%%%%%%%%%%%%%%%%%%%%%%%%%%%
%%% Compilação condicional de capítulos
%\includeonly{}

%%%%%%%%%%%%%%%%%%%%%%%%%%%%%%%%%%%%%%%%%%%%%%%%%%%%%%%%%%%%%%%%%%%%%%%%%%%%%%%%
%%% Começa o documento
\begin{document}

%%%%%%%%%%%%%%%%%%%%%%%%%%%%%%%%%%%%%%%%%%%%%%%%%%%%%%%%%%%%%%%%%%%%%%%%%%%%%%%%
%%% Front matter
\pdfbookmark[1]{LAB: ``Olá, mundo!''}{titulo}
\title{\textbf{LAB: ``Olá, mundo!''}}
\author{Abrantes Araújo Silva Filho}
\date{Revisão: 2023-05-04}
\maketitle
\renewcommand{\abstractname}{\textbf{Resumo}}
\abstract{\noindent Este é um simples LAB de programação em C para verificar se
o seu ambiente está configurado e se você consegue programar, compilar e
executar com sucesso o canônico ``Olá, mundo!'' (o famoso \ingles{Hello,
world!}).}
\pdfbookmark[1]{Sumário}{sumario}
\tableofcontents

%%%%%%%%%%%%%%%%%%%%%%%%%%%%%%%%%%%%%%%%%%%%%%%%%%%%%%%%%%%%%%%%%%%%%%%%%%%%%%%%
%%% Main matter
%\newpage
%%%%%%%%%%%%%%%%%%%%%%%%%%%%%%%%%%%%%%%%%%%%%%%%%%%%%%%%%%%%%%%%%%%%%%%%%%%%%%%%
\section{Introdução}
\label{sec:intro}

O famoso programa ``\textbf{Olá, mundo!}'' é um simples (talvez o mais simples!)
programa de computador que todo estudante/desenvolvedor iniciante escreve quando
começa a aprender uma determinada linguagem de programação.

Em geral o ``Olá, mundo!'' é tão simples que serve basicamente como uma checagem
inicial para verificar se o ambiente de programação está configurado de modo
correto, se as ferramentas de compilação e/ou execução estão instaladas e
funcionando, e se o estudante/desenvolvedor consegue criar um simples código
fonte, transformar esse código fonte em um arquivo executável e rodar esse
código para garantir que ele compreenda o processo e saiba, no mínimo, dar os
primeiros passos na programação.

Neste LAB você programará o famoso ``Olá, mundo!'', utilizando a linguagem C, e
repetirá os primeiros passos percorridos por grandes mestres da computação.


%%%%%%%%%%%%%%%%%%%%%%%%%%%%%%%%%%%%%%%%%%%%%%%%%%%%%%%%%%%%%%%%%%%%%%%%%%%%%%%%
\subsection{Breve histórico}
\label{sec:intro:hist}

Apesar do ``Olá, mundo!'' ser um programa extremamente simples é uma boa
oportunidade para que você aprenda um pouco da história da computação, em
especial o desenvolvimento do Unix\footnote{De acordo com
\citeauthoronline{raymond2004} (\citeyear{raymond2004}), oficialmente o nome
``UNIX'' (com letras maiúsculas) é uma marca registrada do \textbf{The Open
Group} (\url{www.opengroup.org}) e deve ser utilizada apenas para os sistemas
operacionais que foram testados, passaram e foram certificados nos testes e
padrões de conformidade do The Open Group. Ainda de acordo com o mesmo autor, o
nome ``Unix'' (com inicial maiúscula e letras restantes em minúsculas) é
utilizado de modo mais genérico e se refere a qualquer sistema operacional (de
``marca'' Unix ou não) que é um descendente genético do código Unix ancentral
escrito no Bell Lab's, por exemplo os sistemas Linux atuais.} e da Linguagem C.

Nossa jornada pelo tempo começa no final da década de 1960. Martin Richards, em
Cambridge, em 1967, desenvolveu uma linguagem de programação chamada de
\textbf{Basic Combined Programming Language} (BCPL). A BCPL era uma linguagem
muito pequena (rodava em apenas 16k de memória), portável e com uma sintaxe
rica, apesar de não ser uma linguagem tipada: seu único tipo de dados era uma
``machine word'' que podia ser usada como inteiro, caractere, número de ponto
flutuante, ponteiro ou quase qualquer outra coisa dependendo do contexto
\cite{raymond91}.


%%%%%%%%%%%%%%%%%%%%%%%%%%%%%%%%%%%%%%%%%%%%%%%%%%%%%%%%%%%%%%%%%%%%%%%%%%%%%%%%
%%% Apêndices
%\newpage
%\appendix
%\input{apend/xxxx}

%%%%%%%%%%%%%%%%%%%%%%%%%%%%%%%%%%%%%%%%%%%%%%%%%%%%%%%%%%%%%%%%%%%%%%%%%%%%%%%%
%%% Back matter
\bibliography{utils/biblioteca}
%\printindex

%%%%%%%%%%%%%%%%%%%%%%%%%%%%%%%%%%%%%%%%%%%%%%%%%%%%%%%%%%%%%%%%%%%%%%%%%%%%%%%%
%%% Termina o documento
\end{document}
